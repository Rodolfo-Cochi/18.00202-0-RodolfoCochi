\documentclass{article}
\usepackage[brazil]{babel}
\usepackage[utf8]{inputenc}
\usepackage{amsmath}
\usepackage{amsthm}
\usepackage{amssymb}
\usepackage{natbib}
\usepackage{graphicx}
\usepackage{shortvrb}
\usepackage{etoolbox}
\AtBeginEnvironment{align}{\setcounter{equation}{0}}

\title{
    \textbf{Tópicos Avançados em\\Estrutura de Dados\\}
    \medskip
    Atividade 1
}
\author{
\begin{tabular}{l r}
    [NOME]   & [RA]
\end{tabular}
}
\date{10/03/2020}




\begin{document}

\maketitle
\pagebreak

% 1 =======================================================
\section*{Questão 1}

% a =======================================================
\noindent
\textbf{a)}
\begin{align}
    int\ x = 30                  & \longrightarrow \sigma rec + \sigma arm = \textbf{2} \\
    int\ i = 0;                  & \longrightarrow \sigma rec + \sigma arm = \textbf{2} \\
    i < n;                       & \longrightarrow 2 * \sigma rec + \sigma op = \textbf{3(n+1)} \\
    ++i;                         & \longrightarrow 2 * \sigma rec + \sigma op + \sigma arm = \textbf{4n} \\ 
    x = x + 2 - i                & \longrightarrow 3 * \sigma rec + 2 * \sigma op + \sigma arm = \textbf{6} \\
    Formula                      & \longrightarrow  \textbf{$\longrightarrow$ 7n + 13}
\end{align}

% b =======================================================
\noindent
\textbf{b)}
\begin{align}
    int\ abc = 30; & \longrightarrow \sigma arm + \sigma rec = \textbf{2} \\
    for & \nonumber \\
    int i=1 & \longrightarrow \sigma rec + \sigma arm \\
    i<n-1 & \longrightarrow (n-1)*(3 \sigma rec + \sigma op + \sigma arm) \\
    i++ & \longrightarrow (n-2)*(2 \sigma rec + \sigma op + \sigma arm)
    abc\ *= 2; & \longrightarrow 2 \sigma rec + \sigma op + \sigma arm = \textbf{4} \\
    abc++; & \longrightarrow 2 \sigma rec + \sigma op + \sigma arm = \textbf{4} \\
    \textbf{Fórmula } & \textbf{$\longrightarrow$ 9n - 3} 
\end{align}

\medskip
% c =======================================================
\noindent
\textbf{c)}
\begin{align}
    int\ x = 30    & \longrightarrow\ \sigma arm + \sigma rec = \textbf{2} \\ 
    int\ i = 0     & \longrightarrow\ \sigma arm + \sigma rec = \textbf{2} \\
    while (i < n) & \longrightarrow\ n (\sigma rec + \sigma op + \sigma rec) = \textbf{3n} \\
    x = x + 2 - i & \longrightarrow\ \sigma arm + \sigma rec + \sigma op + \sigma rec \sigma rec + \sigma op = \textbf{6} \\
    i = i + 1     & \longrightarrow\ \sigma arm + \sigma rec + \sigma op + \sigma rec = \textbf{4} \\
    \textbf{Fórmula } & \textbf{$\longrightarrow$ 14 + 3n}
\end{align}

% d =======================================================
\noindent
\textbf{d)}
\begin{align}
    int\ abc = 30;   & \longrightarrow \sigma rec + \sigma arm = \textbf{2} \\
    int\ i = 1;      & \longrightarrow \sigma rec + \sigma arm = \textbf{2} \\
    abc\ *= 2;       & \longrightarrow 2 \sigma rec + \sigma op + \sigma arm = \textbf{4(n-2)} \\
    abc++;          & \longrightarrow 2 \sigma rec + \sigma op + \sigma arm = \textbf{4(n-2)} \\
    i = i + 1;      & \longrightarrow 2 \sigma rec + \sigma op + \sigma arm = \textbf{4(n-2)} \\
    while(i < n-1); & \longrightarrow 3 \sigma rec + 2 \sigma op = \textbf{5(n - 3)} \\
    \textbf{Fórmula } & \textbf{$\longrightarrow$ 17n - 35}
\end{align}

% 2 =======================================================
\pagebreak
\section*{Questão 2}

\textbf{\indent Primeira linha:} $int\ resultado = a[n-1]$

\noindent
$\sigma rec (a) + \sigma rec (n) + \sigma rec (1) + \sigma subt + \sigma rec (a[n-1]) + \sigma . (calcular o endereco) + \sigma arm (em\ int\ resultado);$
\newline \indent
7 operações.
\\

\textbf{Segunda linha:} $for(i = 0;\ i < a.length;\ i++)$

\noindent
$\sigma rec (0) + \sigma arm (em i);$

\noindent
$(a.length + 1) * (\sigma rec (i)  + \sigma rec (a.length) + \sigma <);$

\noindent
$(a.length) * (\sigma rec (i) + \sigma rec (1) + \sigma soma + \sigma arm (em i));$
\newline \indent
5 + 7 (a.length) operações.
\\

\textbf{Terceira linha:} $resultado = resultado * x + a[i]$

\noindent
$ (a.length)*(\sigma rec (resultado) + \sigma rec (x) + \sigma mult + \sigma rec (a) + \sigma rec (i) + \sigma . (calculo do endereco) + \sigma rec (a[i]) + \sigma soma + \sigma arm (em resultado));$
\newline \indent
9 (a.length) operações.

\dotfill

\textbf{Operações no total: 12 + 16 (a.length)}




\end{document}
